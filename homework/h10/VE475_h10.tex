%-%-%-%-%-%-%-%-%-%-%-%-%-%-%-%-%-%-%-%-%-%-%-%-%-%-%-%-%-%-%-%-%-%-%-%-%-%-%
%This is a blank document for homework assignments.

%Some preliminaries:  Anything after a '%' is a comment - it isn't read by 
%the compiler.  

%You are welcome to skip down to lines 38-44 to put in some information, and 
%then to line 57 to start writing, but the preamble contains all the 
%formatting that makes it look nice, if you're interested in how that works.

%Packages are just collections of commands to do different things.  For
%almost anything you might want to do, there's a package that will do it.
%-%-%-%-%-%-%-%-%-%-%-%-%-%-%-%-%-%-%-%-%-%-%-%-%-%-%-%-%-%-%-%-%-%-%-%-%-%-%


\documentclass[12pt]{article}  
%The article class is a very basic type of document for writing
%We will customize it to do what we want.

\usepackage[margin=1in]{geometry}  %Adjust margins, formatting

\usepackage{amsmath}  
\usepackage{amssymb}  
\usepackage{amsfonts}  
%These packages add commands for useful symbols and fonts and things like that.
%Most of the time, these are all you need.
\usepackage{epsfig,graphicx,subfigure}
\usepackage{textcomp, gensymb}  %Gives more symbols, like degree
\usepackage{minted}
\usepackage{amsthm}

\usepackage{algorithm}
\usepackage{algorithmic}
\usepackage{fancyhdr}  %Header and Footer formatting
\pagestyle{fancy}  
\renewcommand{\headrulewidth}{0.4pt}
\renewcommand{\footrulewidth}{0.4pt}
\setlength{\headheight}{18pt}

%Header and Footer Information
\lhead{\large{\bf Liqin Zhang 517370910123}}  %Replace with your name
\chead{}
\rhead{\textsc{VE475 h10}}  %Replace "Title" with the name of the assignment
\lfoot{\today}  %You can let it put in today's date or put one in yourself
\cfoot{}
\rfoot{\thepage\ of \ref{NumPages}}  %Counts the pages.

\makeatletter        %This provides a total page count as \ref{NumPages}                 
\AtEndDocument{\immediate\write\@auxout{\string\newlabel{NumPages}{{\thepage}}}}
\makeatother

\usepackage{amsthm}  %This will create the Problem environment
\theoremstyle{definition}
\newtheorem{problem}{Problem}

\begin{document}
\begin{problem}
Group structure on an elliptic curve
\end{problem}
Given an elliptic curve of equation $y^2=x^3+bx+c$, we have two point $P_1=(x_1,y_1),P_2=(x_2,y_2)$.
Based on the defined addition, we have $P_3 = (x_3,y_3) = P_1+P_2$. We will first prove the proposition that states 
$$x_3=m^2-x_1-x_2 \quad y_3 = m(x_1-x_3)-y_1$$
by checking whether $(x_3,y_3)$ fits the elliptic curve function $y_3^2=x_3^3+bx_3+c$.

\begin{align*}
y_3^2&=m^2(2x_1+x_2-m^2)^2-2m(2x_1+x_2-m^2)y_1+y_1^2\\
&=m^6-4m^4x_1-2m^4x_2+2m^3y_1+4m^2x_1x_2+m^2x_2^2-4mx_1y_1-2mx_2y_1
+y_1^2
\end{align*}
\begin{align*}
x_3^3+bx_3+c&=(m^2-x_1-x_2)^3+b(m^2-x_1-x_2)+c\\
&=m^6-3m^4x_1-3m^4x_2+3m^2x_1^2+6m^2x_1x_2+3m^2x_2^2+bm^2\\
&\quad-x_1^3-3x_1^2x_2-3x_1x_2^2-bx_1-x_2^3-bx_2+c
\end{align*}
We need to make sure the difference between the above two terms is 0.
\begin{align*}
x_3^3+bx_3+c-y_3^2&=m^4x_1-m^4x_2-2m^3y_1-m^2x_1^2+2m^2x_1x_2+2m^2x_2^2+bm^2\\
&\quad+4mx_1y_1+2mx_2y_1-x_1^3-3x_1^2x_2-3x_1x_2^2-bx_1-x_2^3-bx-2-y_1^2+c
\end{align*}
\textbf{Case 1.} $P_1\neq P_2$, $m=\dfrac{y_2-y_1}{x_2-x_1}$
    
\begin{align*}
x_3^3+bx_3+c-y_3^2&=
-\frac{1}{(x_1-x_2)^3}(x_1^6 - 3x_1^4x_2^2 + bx_1^4 - 2bx_1^3x_2 - 2x_1^3y_1^2 + 2x_1^3y_1y_2 + x_1^3y_2^2 - cx_1^3 \\
&\quad + 3x_1^2x_2^4 - 3x_1^2x_2y_2^2 + 3cx_1^2x_2 + 2bx_1x_2^3 + 3x_1x_2^2y_1^2 - 3cx_1x_2^2 - bx_1y_1^2 + 2bx_1y_1y_2 \\
&\quad  - bx_1y_2^2- x_2^6 - bx_2^4 - x_2^3y_1^2 - 2x_2^3y_1y_2 + 2x_2^3y_2^2 + cx_2^3 + bx_2y_1^2 - 2bx_2y_1y_2 + bx_2y_2^2 \\
&\quad + y_1^4 - 2y_1^3y_2 + 2y_1y_2^3 - y_2^4)\\
&=-\frac{1}{(x_1-x_2)^3}[
x_1^3(x_2^3 + bx_2 + c) - x_2^3(x_1^3 + bx_1 + c) - 2x_1^3(x_1^3 + bx_1 + c)\\
&\quad + 2x_2^3(x_2^3 + bx_2 + c) + 3x_1^2x_2^4 - 3x_1^4x_2^2 + (x_1^3 + bx_1 + c)^2 - (x_2^3 + bx_2 + c)^2\\
&\quad + bx_1^4 - bx_2^4 - cx_1^3 + cx_2^3 + x_1^6 - x_2^6 - bx_1(x_1^3 + bx_1 + c) + bx_2(x_1^3 + bx_1 + c)\\
&\quad - bx_1(x_2^3 + bx_2 + c) + bx_2(x_2^3 + bx_2 + c) + 2bx_1x_2^3 - 2bx_1^3x_2 - 3cx_1x_2^2 + 3cx_1^2x_2\\
&\quad + 2y_1y_2(x_2^3 + bx_2 + c) + 2x_1^3y_1y_2 - 2x_2^3y_1y_2 + 3x_1x_2^2(x_1^3 + bx_1 + c)\\
&\quad - 3x_1^2x_2(x_2^3 + bx_2 + c) - y_1y_2(2x_1^3 + 2bx_1 + 2c) + 2bx_1y_1y_2 - 2bx_2y_1y_2]\\
&=0
\end{align*}
\textbf{Case 2.} $P_1=P_2$, $m=\dfrac{3x_1^2+b}{2y_1}$
\begin{align*}
x_3^3+bx_3+c-y_3^2&=x_1^3 + bx_1 - y_1^2 + c    \\
&= x_1^3 + bx_1 - (x_1^3+bx+c) + c \\
&=0
\end{align*}

\begin{itemize}
    \item \textbf{Commutative Law}\\
    Suppose $P_1+P_2=(x,y)$, $P_2+P_1=(x',y')$, we want to show $x=x',y=y'$\\
    1. When $P_1=P_2$, it is obviously true.\\
    2. When $P_1 \neq P_2$, $m=m'=\dfrac{y_2-y_1}{x_2-x_1}$.
    $$x=x'=m^2-x_1-x_2$$
    $$y=m(x_1-x)-y_1=\frac{(x_1-x)(y_2-y_1)-(x_2-x_1)y_1}{x_2-x_1}=\frac{x_1y_2-x_2y_1-x(y_2-y_1)}{x_2-x_1}$$
    $$y'=m(x_2-x)-y_2=\frac{(x_2-x)(y_2-y_1)-(x_2-x_1)y_2}{x_2-x_1}=\frac{x_1y_2-x_2y_1-x(y_2-y_1)}{x_2    -x_1}=y$$
    \item \textbf{Associative Law}\\
    Suppose we have points $P_1,P_2,P_3$ on curve $E$. According to the definition, we have 
    $$(P_1+P_2+P_3) = \mathcal{O}$$
    Therefore, we have 
    $$P_1+P_2=-P_3$$
    $$P_2+P_3=-P_1$$
    Thus
    $$(P_1+P_2)+P_3=(-P_3)+P_3=\mathcal{O}$$
    $$P_1+(P_2+P_3)=P_1+(-P_1)=\mathcal{O}$$
    Hence we prove that $(P_1+P_2)+P_3=P_1+(P_2+P_3)$.
\end{itemize}
\begin{problem}
Number of points on an elliptic curve
\end{problem}
\begin{enumerate}
    \item Given $x_1=8,y_1=9,b=3$
    $$m_2\equiv\frac{3x_1^2+3}{2y_1}\equiv9\mod11$$
    $$x_2\equiv m_2^2-2x_1\equiv10\mod11$$
    $$y_2\equiv m_2(x_1-x_2)-y_1\equiv6\mod11$$
    $$[2]P=(10,6)$$
    
    $$m_4\equiv\frac{3x_2^2+3}{2y_2}\equiv6\mod11$$
    $$x_4\equiv m_4^2-2x_2\equiv5\mod11$$
    $$y_4\equiv m_4(x_4-x_4)-y_2\equiv2\mod11$$
    $$m_5\equiv\frac{y_4-y_1}{x_4-x_1}\equiv6\mod11$$
    $$x_5\equiv m_5^2-x_4-x_1\equiv1\mod11$$
    $$y_5\equiv m_5(x_4-x_5)-y_4\equiv0\mod11$$
    $$[5]P=(1,0)$$
    
    $$m_{10}\equiv\frac{3x_5^2+3}{2y_5}\text{ dosen't exist.}$$
    $$[10]P=\mathcal{O}=(0,0)$$
    \item About 11. (10 according to (3).)
    \item Given elliptic curve $y^2=x^3+3x+7 \text{ mod }11$. The points on $E$ are pair of elements $(x,y)$ that satisfy the equation.
    \begin{center}
    \begin{tabular}{cccc}
    \hline
    $x\mod11$ & $y^2\mod11$ & $y\mod11$ & Points on $E$\\\hline
    0  & 7  & \\
    1  & 0  & 0 & (1,0)\\
    2  & 10 & \\
    3  & 10 & \\
    4  & 6  &  \\
    5  & 4  & 2 or 9 & (5,2) or (5,9) \\
    6  & 10 & \\
    7  & 8  & \\
    8  & 4  & 2 or 9 & (8,2) or (8,9) \\
    9  & 4  & 2 or 9 & (9,2) or (9,9) \\
    10 & 3  & 5 or 6 & (10,5) or (10,6) \\\hline
    \end{tabular}
    \end{center}
    Including the infinite point $\mathcal{O}$, there are 10 points in total.
\end{enumerate}

\begin{problem}
ECDSA
\end{problem}

In the Elliptic Curve Digital Signature Algorithm (ECDS), we need a cryptographic hash function $h$, an elliptic curve $E$, a Point $G\in E$, the order $n$ of $G$ such that $[n]G=\mathcal{O}$.

\textbf{Initial setup}
\begin{enumerate}
    \item Creates a key pair, consisting of a private key integer $d_{A}$
    \item Randomly selected in the interval $[1,n-1]$
    \item a public key curve point $Q_A=[d_A]G$
\end{enumerate}

\textbf{Sign procedure}
\begin{enumerate}
\item Calculate $e=h(m)$.
\item Let $z$ be $L_n$ leftmost bits of $e$, where $L_n$ is the  bit length of the group order $n$.
\item Generate a random integer $k$ in $[1,n-1]$.
\item Calculate $P:(x_1,y_1)=[k]G$.
\item Calculate $r\equiv x_1\mod n$. If $r=0$, retry from step 3.
\item Calculate $s\equiv k^{-1}(z+rd_A)\mod n$. If $s=0$, retry from step 3.
\item The signature is the pair $(r,s)$.
\end{enumerate}

\textbf{Authentication procedure}
\begin{enumerate}
\item Check that $Q_{A}$ is not equal to the identity element $\mathcal{O}$.
\item Check that $Q_{A}$ lies on the curve.
\item Check that $[n]Q_{A}=\mathcal{O}$.
\item Verify that $r$ and $s$ are integers in $[1,n-1]$. If not, the signature is invalid.
\item Calculate $e=h(m)$.
\item Let $z$ be $L_n$ leftmost bits of $e$, where $L_n$ is the  bit length of the group order $n$.
\item Calculate $w\equiv s^{-1}\mod n$
\item Calculate $u_1\equiv zw\mod n$ and $u_2\equiv rw\mod n$.
\item Calculate the curve point $P:(x_1,y_1)=[u_1]G+[u_2]Q_A$. If $P=\mathcal{O}$, the signature is invalid.
\item The signature is valid if $r\equiv x_1\mod n$, invalid otherwise.
\end{enumerate}

\textbf{Validation}
\begin{align*}
P&=[u_1]G+[u_2]Q_A\\
&=[u_1+u_2d_A]G\\
&=[zs^{-1}+rd_As^{-1}]G\\
&=[(z+rd_A)s^{-1}]G\\
&=[(z+rd_A)(k^{-1}(z+rd_A))^{-1}]G\\
&=[k]G
\end{align*}

\textbf{Benefits}
\begin{itemize}
    \item ECDSA is about twice the size of the security level, which saves a lot of key bits compared to other algorithms like DSA.
    \item ECDSA has faster algorithms for generating signatures because of the computation involves smaller numbers.
    \item ECDSA has a smaller size of data of certificate to establish a TLS connection which leads to faster connection.
\end{itemize}

\begin{problem}
BB84
\end{problem}
BB84 is the first quantum cryptography protocol, which is provably secure, relying on the quantum property that information gain is only possible at the expense of disturbing the signal if the two states one is trying to distinguish are not orthogonal and an authenticated public classical channel.

In the BB84 scheme, Alice wishes to send a private key to Bob. She begins with two strings of bits, $a$ and $b$, each $n$ bits long. She then encodes these two strings as a string of $n$ qubits:
$$|\psi\rangle=\bigotimes_{i=1}^n|\psi_{a_ib_i}\rangle$$
where $a_i$ and $b_i$ are i-th bit of a and b. Together $a_ib_i$ provides the index of the following four qubits.
$$|\psi_{00}\rangle=|0\rangle$$
$$|\psi_{10}\rangle=|1\rangle$$
$$|\psi_{01}\rangle=\frac{1}{\sqrt{2}}|0\rangle+\frac{1}{\sqrt{2}}|1\rangle$$
$$|\psi_{11}\rangle=\frac{1}{\sqrt{2}}|0\rangle-\frac{1}{\sqrt{2}}|1\rangle$$

Alice sends $|\psi \rangle$  over a public and authenticated quantum channel $\mathcal {E}$ to Bob. Since only Alice knows $b$, it makes it virtually impossible for either Bob or Eve to distinguish the states of the qubits. Also, after Bob has received the qubits, we know that Eve cannot be in possession of a copy of the qubits sent to Bob, by the no-cloning theorem, unless she has made measurements. Her measurements, however, risk disturbing a particular qubit with probability 1/2 if she guesses the wrong basis.

Bob proceeds to generate a string of random bits $b'$ of the same length as $b$ and then measures the string he has received from Alice, $a'$. At this point, Bob announces publicly that he has received Alice's transmission. Alice then knows she can now safely announce $b$. Bob communicates over a public channel with Alice to determine which $b_{i}$ and $b'_{i}$ are not equal. Both Alice and Bob now discard the qubits in $a$ and $a'$ where $b$ and $b'$ do not match.

From the remaining $k$ bits where both Alice and Bob measured in the same basis, Alice randomly chooses $k/2$ bits and discloses her choices over the public channel. Both Alice and Bob announce these bits publicly and run a check to see whether more than a certain number of them agree. If this check passes, Alice and Bob proceed to use information reconciliation and privacy amplification techniques to create some number of shared secret keys. Otherwise, they cancel and start over.

\begin{problem}
Quantum key distribution
\end{problem}
\begin{enumerate}
    \item
Alice and Bob could use the quantum channel to distribute a quantum key according to some protocols like BB84, and they use the classic channel to send message encrypted by that key.
\item
During the key distribution in quantum channel, Eve's measurements towards the random elements will lead to errors found when comparing keys. If Eve chooses the same base as Alice, it will not affect Bob's measurement results, and Alice and Bob will not find Eve when they compare part of the key. But there is still a 50\% probability that Eve will choose a different basis from Alice to make the measurement, which will change the state of that element. At this time, Bob will has his measurement again with a 50\% probability to get a different result from Alice. Therefore they can easily find out the existence of the Eve when the comparing quantum states are different, and they could immediately change another undistributed key to use on the classical channel.

\end{enumerate}
\begin{problem}
Simple questions
\end{problem}
\begin{enumerate}
\item Given $n \times n$ matrices $U_1,U_2,V_1,V_2$, denote 
\begin{align*}
U_1&=
\begin{pmatrix}
u_{1,1,1} & u_{1,1,2} & \cdots & u_{1,1,n} \\
u_{1,2,1} & u_{1,2,2} & \cdots & u_{1,2,n} \\
\vdots & \vdots & \ddots & \vdots \\
u_{1,n,1} & u_{1,n,2} & \cdots & u_{1,n,n} \\
\end{pmatrix}
\end{align*}
And $U_2,V_1,V_2$ shares the same notation as above. 

According to the definition of \textbf{kronecker product,} we have
\begin{align*}
(U_1\otimes V_1)(U_2\otimes V_2)&=
\begin{pmatrix}
u_{1,1,1}V_1 & \cdots & u_{1,1,n}V_1 \\
\vdots &  \ddots & \vdots \\
u_{1,n,1}V_1 & \cdots & u_{1,n,n}V_1 \\
\end{pmatrix}
\cdot
\begin{pmatrix}
u_{2,1,1}V_2 &\cdots & u_{2,1,n}V_2 \\
\vdots &  \ddots & \vdots \\
u_{2,n,1}V_2  & \cdots & u_{2,n,n}V_2 \\
\end{pmatrix}
\\&=
\begin{pmatrix}
\sum_{i=1}^nu_{1,1,i}u_{2,i,1}V_1V_2 & \cdots & \sum_{i=1}^nu_{1,1,i}u_{2,i,n}V_1V_2 \\
\vdots &  \ddots & \vdots \\
\sum_{i=1}^nu_{1,n,i}u_{2,i,1}V_1V_2 & \cdots & \sum_{i=1}^nu_{1,n,i}u_{2,i,n}V_1V_2 \\
\end{pmatrix}
\\&=U_1U_2 \otimes V_1V_2
\end{align*}
\item To show the operator $\otimes$ is bilinear, which means we need to show 
$$U_1 \otimes (V_1 + V_2) = U_1 \otimes V_1 +  U_1 \otimes V_2 $$
$$(U_1 + U_2)\otimes V_1  = U_1 \otimes V_1 +  U_2 \otimes V_1 $$
I will only show the detailed proof for the first equation because they are very similar.
\begin{align*}
U_1 \otimes (V_1 + V_2)&=
\begin{pmatrix}
u_{1,1,1}(V_1+V_2) & \cdots & u_{1,1,n}(V_1+V_2) \\
\vdots &  \ddots & \vdots \\
u_{1,n,1}(V_1+V_2) & \cdots & u_{1,n,n}(V_1+V_2) \\
\end{pmatrix}
\\&=
\begin{pmatrix}
u_{1,1,1}V_1 & \cdots & u_{1,1,n}V_1 \\
\vdots &  \ddots & \vdots \\
u_{1,n,1}V_1 & \cdots & u_{1,n,n}V_1 \\
\end{pmatrix}
+
\begin{pmatrix}
u_{1,1,1}V_2 & \cdots & u_{1,1,n}V_2 \\
\vdots &  \ddots & \vdots \\
u_{1,n,1}V_2 & \cdots & u_{1,n,n}V_2 \\
\end{pmatrix}
\\&=U_1 \otimes V_1 +  U_1 \otimes V_2
\end{align*}
Hence, $\otimes$ is bilinear.
\end{enumerate}

\end{document}